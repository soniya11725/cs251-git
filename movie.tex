\documentclass[11pt]{article}
\begin{document} 

\title{Movie: Anand}
\author{Soniya Barmate,Sanket Yadav}
\date{April 2,2017}
\maketitle
\section*{Anand}
\subsection*{Introduction}
Anand (English: Bliss) is a 1971 Indian drama film co-written and directed by Hrishikesh Mukherjee.
It stars Rajesh Khanna in the lead role with and supporting cast includes Amitabh Bachchan, Sumita Sanyal, 
Ramesh Deo, Seema Deo. Khanna played the title role. The dialogues were written by Gulzar. The film won 
several awards including the Filmfare award for best film in 1972. In 2013, it was listed in Anupama Chopra's
book "100 Films To See before You Die".[1]This film is counted among the 17 consecutive hit films Rajesh Khanna
between 1969 and 1971, by adding the two hero films Marayada and Andaz to the 15 consecutive solo hits he gave 
from 1969 to 1971.
\subsection*{Plot}
The film begins with a felicitation ceremony arranged for Bhaskar (Amitabh Bachchan), a doctor who has just 
written a successful book titled Anand. Bhaskar was a cancer specialist and after the congratulatory 
speeches, he reveals that the book is not a work of fiction but taken from his own diary and pertains
to his experiences with a real person named Anand.

Flashback starts with Bhaskar, little fresh from his training as an oncologist, trying to treat the 
poor for no charge but often gets disheartened by the fact that he cannot cure all the ailments in
the world. He becomes pessimistic after seeing the suffering, illness and poverty all around him.
He acts straightforward and would not treat the imaginary ailments of the rich. But his friend,
Kulkarni follows a little different path. He treats the imaginary illnesses of the rich and uses
that money to treat the poor.

One day Kulkarni introduces Bhaskar to Anand (Rajesh Khanna) who has lymphoma of the intestine, 
a rare type of cancer. Anand has a cheerful nature and despite knowing the truth that he is not
going to survive for more than six months, he maintains a nonchalant demeanour and always tries
to make everyone happy around him. His cheerful and vibrant nature soothes Bhaskar, who has a 
contrasting nature and they become good friends.

Anand's condition gradually deteriorates but he does not want to spend his remaining time in the 
hospital bed; he instead roams freely and helps everyone. He discovers that Bhaskar has strong 
feelings for Renu (Sumita Sanyal), whom he treated previously for pneumonia. He helps Bhaskar to
express his love and convinces Renu's mother for their marriage. He tells Bhaskar that everyone
should remember him as a lively person and not as a cancer patient. His end comes and he dies 
amongst his friends and everyone remembers him as a vibrant and lively person. Bhaskar becomes
more philosophical and continues to help the helpless with more empathy and maturity.
\end{document}
