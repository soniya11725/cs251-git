\documentclass[11pt]{article}
\begin{document} 

\title{Movie reviews}
\author{Soniya Barmate\\Sanket Yadav}
\date{April 2,2017}
\maketitle

\section*{3 idiots}
\subsection*{Review}
Rest assured, all ye desi cinema buffs, Aal Izz Well in apna Bollywood. If 2009 can begin with Dev D and end with 3 Idiots, it is indeed time to sound the seetis and taalis for one of the most exciting years of contemporary Indian cinema. Truly, this has been the year of the I.d.i.o.t in movielore: the Intrinsically intelligent, Downright smart, Inimitable, Original and Talented film maker, actor, story teller, musician, lyricist, dialogue writer and producer.
3 Idiots is the perfect end to an exciting year for India: the year when the aam aadmi voted in progress, liberalism, secularism and turned his back to corruption, communalism, regionalism. The three idiots, Rancchoddas Shyamaldas Chanchad (Aamir Khan), Raju Rastogi (Sharman Joshi) and Farhan Qureshi (R Madhavan), are perfect archetypes of the new age Indian who is essentially a non-conformist, questioning outmoded givens, choosing to live life on his own terms and chartering new roads that consciously skirt the rat race. Of course, they begin on the beaten track -- due to societal/parental pressure -- but refuse to become cogs in the wheel. Naturally, they end up as the Frostian hero (Robert Frost's Road Not Taken) who made all the difference to his life, and the world, by taking the road less travelled by.

The film begins with the entry of our threesome in the city's elite engineering college. It takes the first tryst with the mandatory ragging sessions which enunciate who the leader of the gang is going to be: new entrant Baba Rancchoddas, as his friends fondly call him. Rancho not only leads his friends through the maze of India's competitive, high-pressure, rote-heavy, illogical and almost cruel education system, he tutors them on several life mantras too. Like, running after excellence, not success; questioning not blindly accepting givens; inventing and experimenting in lieu of copying and cramming; and essentially following your heart's calling if you truly want to make a difference.

So, you have the threesome embroiled, time and again, in a confrontation with authority, as represented through the domineering figure of Viru Sahastrabuddhe (Boman Irani), the unsmiling Principal who venerates the cuckoo because the bird's life begins with murder. Kill the competition, because there is only one place at the top, believes the Princi. Poor, mistaken Princi! Doesn't he know that competition is effete, model students like Chatur (Omi) end up as duhs in real life and non-conformists (Rancho and Rocket Singh Inc), who care tuppence about being on top, could end up as eventual winners. More importantly, they could be high not only in IQ (intelligence quotient) but in EQ (emotional quotient) too, never losing their humaneness and social networking skills.

The high point of the film is the fact that director Rajkumar Hirani says so much, and more, without losing his sense of humour and the sheer lightness of being. The film is a laugh riot, despite being high on fundas. Certain sequences almost have you rolling in the aisle, like the ragging sequence, Omi's chamatkar/balatkar speech, the threesome's wedding crasher sequence, their mournful meal with Raju's mournful mum and Rancho's sundry demos to prove how Kareena has chosen the wrong guy for herself. Add to this, the strong emotional core of the film that makes gentle tugs, now and then, at your guts, and you have an almost perfect score. Hirani carries forward his simplistic `humanism alone works' philosophy of the Lage Raho Munnabhai series in 3 Idiots too, making it a warm and vivacious signature tune to 2009. The second half of the film does falter in parts, specially the child birth sequence, but it doesn't take long for the film to jump back on track.

Amongst the performances, Aamir Khan is stupendous as the rule-breaker Rancho. But the rest of the cast doesn't remain in the shadows. Both Sharman and Madhavan manage to carve their independent characters as lovable rebels too. Even Kareena shines out, despite the minuscule length of her role. A special mention for Boman Irani who is impeccable as `Virus', the vile Principal and newcomer Omi who perfectly slips into the stereotype of the best, albeit bakwas student. Shantanu Moitra's music score, which may have sounded pheeka in the audio version, comes alive on screen with lyricist Swanand Kirkire giving India its clarion call for 2010: Aal Izz Well. Rush for it.

A word about:

Performances: Believe it or not, but Aamir, Madhavan and Sharman actually look -- and behave -- like students. While Aamir pitches in a near-perfect portrayal of Rancho, the free-spirited innovator, Madhavan and Sharman are perfectly in sync too. Kareena as the independent-minded medical student is winsome; debutant Omi has a refreshing flair for comedy and Boman Irani doesn't ham or go over the top even once.

Story: Rajkumar Hirani and Abhijat Joshi script a warm and humanist indictment of India's rude-crude education system that prepares rats for a rat race rather than thinkers for a new world.

Dialogue: Witty and wild, the film walks away with the best comic scene of the year citation with its uproarious `balatkar' speech.

Music: Shantanu Moitra may not have forced you to pick up the music album of the film but the songs do come alive on screen, specially Zoobie-Doobie and Aal Izz Well.

Choreography: Avit Diaz has the threesome -- Aamir, Madhavan, Sharman -- kick up some wild fun in Aal Izz Well, while Bosco-Caesar rightly go retro with Zoobie-Doobie.

Cinematography: The streets of Delhi and the picture postcard beauty of Ladakh are captured in riveting images by Muraleedharan CK

Styling: Designers Manish Mehrotra, Sheena Parekh and Raghuveer Shetty create the pucca campus look for our rumbustious kids on the block, complete with ganjis and capris. Kareena too is an archetypal Dilli gal with her trendy, not flashy ensemble.

Inspiration: Chetan Bhagat's Five Point Someone literally comes alive on screen, although the film rocks and chetan bhagat sucks.
\subsection{This is just for timepass, just kidding}
Man I hate group assignmets.
\section*{Anand}
\subsection*{Introduction}
Anand (English: Bliss) is a 1971 Indian drama film co-written and directed by Hrishikesh Mukherjee.
It stars Rajesh Khanna in the lead role with and supporting cast includes Amitabh Bachchan, Sumita Sanyal, 
Ramesh Deo, Seema Deo. Khanna played the title role. The dialogues were written by Gulzar. The film won 
several awards including the Filmfare award for best film in 1972. In 2013, it was listed in Anupama Chopra's
book "100 Films To See before You Die".[1]This film is counted among the 17 consecutive hit films Rajesh Khanna
between 1969 and 1971, by adding the two hero films Marayada and Andaz to the 15 consecutive solo hits he gave 
from 1969 to 1971.
\subsection*{Plot}
The film begins with a felicitation ceremony arranged for Bhaskar (Amitabh Bachchan), a doctor who has just 
written a successful book titled Anand. Bhaskar was a cancer specialist and after the congratulatory 
speeches, he reveals that the book is not a work of fiction but taken from his own diary and pertains
to his experiences with a real person named Anand.

Flashback starts with Bhaskar, little fresh from his training as an oncologist, trying to treat the 
poor for no charge but often gets disheartened by the fact that he cannot cure all the ailments in
the world. He becomes pessimistic after seeing the suffering, illness and poverty all around him.
He acts straightforward and would not treat the imaginary ailments of the rich. But his friend,
Kulkarni follows a little different path. He treats the imaginary illnesses of the rich and uses
that money to treat the poor.

One day Kulkarni introduces Bhaskar to Anand (Rajesh Khanna) who has lymphoma of the intestine, 
a rare type of cancer. Anand has a cheerful nature and despite knowing the truth that he is not
going to survive for more than six months, he maintains a nonchalant demeanour and always tries
to make everyone happy around him. His cheerful and vibrant nature soothes Bhaskar, who has a 
contrasting nature and they become good friends.

Anand's condition gradually deteriorates but he does not want to spend his remaining time in the 
hospital bed; he instead roams freely and helps everyone. He discovers that Bhaskar has strong 
feelings for Renu (Sumita Sanyal), whom he treated previously for pneumonia. He helps Bhaskar to
express his love and convinces Renu's mother for their marriage. He tells Bhaskar that everyone
should remember him as a lively person and not as a cancer patient. His end comes and he dies 
amongst his friends and everyone remembers him as a vibrant and lively person. Bhaskar becomes
more philosophical and continues to help the helpless with more empathy and maturity.

\subsection*{Cast}
Rajesh Khanna as Anand Sehgal
Amitabh Bachchan as Dr. Bhaskar Banerjee a.k.a. Babu Moshai
Rekha as Meera (Special Appearance)
Sumita Sanyal as Renu
Ramesh Deo as Dr. Prakash Kulkarni a.k.a. Dost
Seema Deo as Suman Kulkarni
Lalita Pawar as Matron
Durga Khote as Renu's mother
Johnny Walker as Isa Bhai
Asit Sen as Bhaskar's patient
Dara Singh as Pahalwan




\subsection*{Review}
It is very rare (especially for a man) to shed tears after watching a movie but this
one does make you shed tears for Anand, the main protagonist played by the superstar
of the 1970s, Rajesh Khanna. The movie has everything going for it. Acting, direction,
story, music, dialogues etc, everything is fabulous. It has dollops of drama, humour 
and emotions. It is a story a dying man who looks at life with a positive attitude 
and enjoys his time knowing fully well his disease is incurable and that he is going
to die soon.
Anand Sehgal (Rajesh Khanna) arrives in Mumbai to get treated for his lymph sarcoma 
of the intestine under Dr. Prakash (Ramesh Deo). He encounters the matron (Lalita Pawar)
there and gets scared of her disciplinary attitude. He then befriends Prakash’s colleague,
Dr. Bhaskar Bannerjee (Amitabh) and tries to bring happiness in his life. Anand decides 
to stay with Bhaskar instead of getting admitted in Dr. Prakash’s nursing home. Bhaskar
is a no nonsense character who treats his patients with utmost seriousness and treats his
life also in a similar way. He likes a girl (Sumita Sanyal) whom he treated for Malaria
but is shy of expressing his love to her. The story moves on as Anand continues to befriend
people from across the life. He meets his Murarilal (Johnny Walker) who runs a drama company
of his own and also tries his hand in acting.


\end{document}
